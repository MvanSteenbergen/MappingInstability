%%%%%%%%%%%%%%%%%%%%%%%%%%%%%%%%%%%%%%%%%%%%%%%%%%%%%%%%%%%%%%%%%%%%%%%%%%%%%%%%%%%%%%%%%%%%%%%%%%%%%%%%%%%%%%%%%%%%%%%%%%%%%%%%%%%%%%%%%%%%%%%%%%%%%%%%%%%
% This is just an example/guide for you to refer to when submitting manuscripts to Frontiers, it is not mandatory to use Frontiers .cls files nor frontiers.tex  %
% This will only generate the Manuscript, the final article will be typeset by Frontiers after acceptance.   
%                                              %
%                                                                                                                                                         %
% When submitting your files, remember to upload this *tex file, the pdf generated with it, the *bib file (if bibliography is not\daleth  within the *tex) and all the figures.
%%%%%%%%%%%%%%%%%%%%%%%%%%%%%%%%%%%%%%%%%%%%%%%%%%%%%%%%%%%%%%%%%%%%%%%%%%%%%%%%%%%%%%%%%%%%%%%%%%%%%%%%%%%%%%%%%%%%%%%%%%%%%%%%%%%%%%%%%%%%%%%%%%%%%%%%%%%

%%% Version 3.4 Generated 2022/06/14 %%%
%%% You will need to have the following packages installed: datetime, fmtcount, etoolbox, fcprefix, which are normally inlcuded in WinEdt. %%%
%%% In http://www.ctan.org/ you can find the packages and how to install them, if necessary. %%%
%%%  NB logo1.jpg is required in the path in order to correctly compile front page header %%%

\documentclass[utf8]{FrontiersinVancouver} 
\usepackage{url,hyperref,lineno,microtype,subcaption,framed, float}
\usepackage[onehalfspacing]{setspace}

\linenumbers 
% Leave a blank line between paragraphs instead of using \\ 


\def\keyFont{\fontsize{8}{11}\helveticabold }
\def\firstAuthorLast{van Steenbergen} %use et al only if is more than 1 author
\def\Authors{Maas van Steenbergen\,$^{1,*}$}
% Affiliations should be keyed to the author's name with superscript numbers and be listed as follows: Laboratory, Institute, Department, Organization, City, State abbreviation (USA, Canada, Australia), and Country (without detailed address information such as city zip codes or street names).
% If one of the authors has a change of address, list the new address below the correspondence details using a superscript symbol and use the same symbol to indicate the author in the author list.
\def\Address{$^{1}$ Faculty of Behavioural and Social Sciences,  Methodology \& Statistics, Utrecht University, the Netherlands  }
% The Corresponding Author should be marked with an asterisk
% Provide the exact contact address (this time including street name and city zip code) and email of the corresponding author
\def\corrAuthor{Corresponding Author}

\def\corrEmail{m.vansteenbergen@uu.nl}


\begin{document}
\onecolumn
\firstpage{1}

\title[Recovering Dynamics Using RQA]{Recovering Dynamics of Latent Variables Using Recurrence Quantification Analysis} 

\author[\firstAuthorLast]{\Authors} %This field will be automatically populated
\address{} %This field will be automatically populated
\correspondance{} %This field will be automatically populated

\extraAuth{}% If there are more than 1 corresponding author, comment this line and uncomment the next one.
%\extraAuth{corresponding Author2 \\ Laboratory X2, Institute X2, Department X2, Organization X2, Street X2, City X2 , State XX2 (only USA, Canada and Australia), Zip Code2, X2 Country X2, email2@uni2.edu}

\maketitle

%%% Leave the Abstract empty if your article does not require one, please see the Summary Table for full details.

\section{Introduction}

Intensive longitudinal methods are useful for capturing and analyzing dynamic, real-time variations in individuals' behaviors and experiences over a period of time \citep{bolgerIntensiveLongitudinalMethods2013}. They come with a unique set of methodological challenges\footnote{Note that we avoid the terms `idiographic' and `nomothetic' here. They seem to bring more confusion than clarity. For reasons why this is the case, see Lamiell's work \citep{lamiellNomotheticIdiographicContrasting1998}}. Psychological variables are generally latent: they are not directly observable and our knowledge of their mechanisms is incomplete at best \citep{bollenLatentVariablesPsychology2002}. Therefore, data from latent constructs is usually less precise, as we have to rely on subjective assessment to estimate them \citep{borsboomLatentVariableTheory2008}. Whereas between-person methods rely on averaging out the effects of time and within-person variation to deal with the complications this causes, (quantitative) dynamical within-person methods rely on that variation to make inferences about the underlying trajectory: how the variable fluctuates over time \citep{molenaarManifestoPsychologyIdiographic2004,molenaarNewPersonSpecificParadigm2009,lamiellStatisticalThinkingPsychology2019}. We aim to see whether Recurrence Quantification Analysis (RQA) can be used to recover aspects of the true trajectory from this less precise data \citep{webber2005recurrence}. 

To introduce our topic, we make a number of assumptions about the nature of psychological constructs that are studied using intensive longitudinal methods. It is important to make these assumptions explicit to increase the ability to reject their tenets if they turn out not to hold \citep{meehlTheoreticalRisksTabular2004}. We will use these assumptions to introduce the topic and embed the study in the literature. We note that these assumptions are very close to `common sense' beliefs within the quantitative dynamical within-person research community. To our understanding, however, these assumptions are not made explicit all that often. We invite the reader to evaluate them critically, and even think of arguments or experiments to disprove them. To help this process along, we present some questions that can function as a starting point to think about the validity of these assumptions. 

The first assumption is our working definition of psychological constructs. We take the perspective that these are latent variables that attempt to measure the phenomena of interest in psychology \citep{borsboomLatentVariableTheory2008}. They are not directly observable and our knowledge of these phenomena is incomplete \citep{friedWhatArePsychological2017, maraunAugustinianMethodologicalFamily2009}. As such, one does not know the true value of these constructs, both because of the incapacity to measure them directly and the `error' that comes with measuring them. For example, take the construct of happiness. To measure this construct, we could ask someone how happy they are. There is no external `measuring tape' to judge whether their account of their happiness is equivalent to their `true' state of happiness.

\begin{framed} 
    Questions for assumption one:
    \begin{itemize}
        \item Can you quantify happiness?
        \item What about community spirit?
        \item Can you be happy without knowing it?
        \item Can you make errors when judging your own feelings?
        \item If so, how do we define feelings if we cannot judge them ourselves?
    \end{itemize}
\end{framed}

Secondly, we posit that there is an underlying continuous real-valued trajectory of the features that constructs aim to measure \citep{hamakerNoTimePresent2017}. This follows from the fact that if someone has a `true score', this score needs to be definable even when it cannot be measured, i.e., if somebody is asked to rate themselves, there is a true score that is being approximated using the latent construct. We also assume that that true score changes smoothly over time. It can drop or increase very quickly, but not instantaneously. Further, intensive longitudinal measures are time-dependent. These measures are shaped by different forces and the previous trajectory of the construct \citep{olthofComplexityPsychologicalSelfratings2020b}. They are `complex' measures, meaning that they come to be through the interdependencies of the numerous non-trivially interacting forces that influence the system \citep{olthofComplexityTheoryPsychopathology2023}.

\begin{framed}
    Questions for assumption two:
    \begin{itemize}
        \item How depressed are you when you are asleep?
        \item How agreeable are you when you are very focused on something?
    \end{itemize}
\end{framed}

The last assumption is that ordinal likert-type scales are approximations of this underlying continuous measure \citep{haslbeckRecoveringWithinpersonDynamics2022}. Thus, we assume that there is information loss that comes with these ordinal representations of continuous variables \citep{westlandInformationLossBias2022}. We see this as `error', because we postulate it to be a deviation from its actual, continuous state. For our purposes, we discard other types of error.

\begin{framed}
    Questions for assumption three:
    \begin{itemize}
        \item Did you ever feel there were not enough answering options to answer a likert-type questionnaire fully? (Did you ever feel a 3.5 out of five about your shopping experience?)
        \item Would you have trouble answering a question with too many (ordinal) answer options?
    \end{itemize}
\end{framed}

A technique that stands out for its broad applicability is RQA.\@This method identifies recurrent patterns of time series data \citep{webber2005recurrence}. It results in different indicators for the stability, predictability, and dynamic behavior inherent in these systems. Our research question follows naturally from these assumptions and uses recurrence methods to try to find a solution: if one were to make an explicit theoretical prediction for trajectory, it would be difficult to validate it, as we would have to test our predictions about this continuous trajectory using ordinal measurements of low granularity \citep{haslbeckRecoveringWithinpersonDynamics2022}. We do not have sufficient information to consider the trajectory immediately, but there needs to be an intermediary step where we reconstruct relevant aspects of the trajectory from these ordinal measurements. Our research goal is to find out whether RQA is suitable to fulfill this role. 

We start by using computational methods to simulate each of the three assumptions described above. We then try to infer characteristics of the trajectory from the degraded data using RQA.\@We first simulate a trajectory through dynamical computational modelling \citep{grahekAnatomyPsychologicalTheory2021,gauldDynamicalSystemsComputational2023}, and then break it down by binning the data and removing time points. We use a toy model that simulates symptomatology by Gauld \& Depannemaecker to generate the trajectories \citep{gauldDynamicalSystemsComputational2023}. Symptomatology is a subset of latent variable constructs, which makes it well-suited for our purposes. It allows for the specification of trajectories with a wide range of behaviour. 

The overarching goal is to develop methods to recover aspects of a trajectory empirically using intensive longitudinal studies based on infrequent, low-resolution measurements. While full recovery of trajectory is impossible, it may be possible to recover some relevant aspects of the system under study. The research question is `Given that a psychological construct has a real-valued continuous trajectory, can we recover elements of it using RQA from limited sampling occurrences on an ordinal scale?'. The major elements to be examined include the stability of several recurrence indicators under degradation, the implications for measurement and analysis of time series of latent variable constructs, and the weaknesses and oversights that we found when we tried to simulate a theoretical trajectory and degrade it.  

A strength of this project is that it explicates normally tacit assumptions, and uses these assumptions to model the entire process: we model the underlying trajectory, the latent variable that estimates this trajectory, and we make an explicit prediction for its relationship if these assumptions are met. We use computational methods to generate the data because it is impossible to answer this question in the same way empirically as the real underlying trajectory is unknown. There is, however, an important trade-off being made: because we simulate our data, many of the complicating aspects that would come up during empirical studies are overlooked. We do not use real data, and that means that the inferences are only correct when our assumptions are correct. Serious objections to these assumptions can be found in the discussion section. Another weakness is that the performance of recurrence methods can be sensitive to the parameter settings of the computational model. In the current project, we treat only four different trajectories.

\section{Materials and methods}
\subsection{Stage 1: Data generation}
In the first stage, we used a toy model to simulate the data based on the \textit{3 + 1 Dimensions Model} introduced by \citep{gauldDynamicalSystemsComputational2023}. This model captures clinical observations found in psychiatric symptomatology by modeling internal factors ($y$), environmental noise ($z$), temporal specificities ($f$), and symptomatology ($x$) using coupled differential equations. $x$ is the basis of the time series. The original aim of this toy model is to simulate the trajectory of symptomatology of psychiatric symptoms over time. It is suitable for our project because the model creates realistic looking trajectories for psychological phenomena. The model uses four coupled differential equations to model the effect of time on symptom intensity. Symptom intensity is a subset of latent constructs, and we see its behaviour over time as similar to other latent constructs. It is of note that many different systems could have let to similar results to the ones outputted here. The data generating process is of secondary importance: it should result in somewhat plausible trajectories. We chose this model over more traditional choices, such as the Lorenz attractor, because we prioritized its flexibility in capturing realistic trajectories based on latent variable constructs in the social sciences.

\subsubsection{Symptom intensity}
\begin{equation}
    \tau_{x}\frac{dx}{dt} = \frac{S_{\max}}{1+\exp(\frac{Rs-y}{\lambda_{s}})} - x
\end{equation}

\subsubsection{Modelling of internal elements}
\begin{equation}
    \tau_{y}\frac{dy}{dt} = \frac{P}{1+\exp(\frac{R_{b}-y}{\lambda_{b}})} + L - xy - z
\end{equation}

\subsubsection{Modelling of perceived environment}
\begin{equation}
    \tau_{z}\frac{dz}{dt} = S(ax + \beta y)\ \zeta(t) - z
\end{equation}

\subsubsection{Temporal specificities}
\begin{equation}
    \tau_f\frac{df}{dt} = y - \lambda_f f
\end{equation}

\paragraph{Parameter definitions}
Parameter definitions and parameter settings are shortly mentioned here. For a more in-depth treatment, see Gauld and Depannemaecker \citep{gauldDynamicalSystemsComputational2023}. $\alpha$ \& $\beta$ are the weight of the effect of variables $x$ and $y$ on environmental perception. $\tau_{x,y,z,f}$ are the different time scales the equations operate on. $S_{\max}$ is the maximum level of the symptoms. $R_{s, b}$ is the sensitivity to triggering the system.  $\lambda_{s,b}$ are the slopes of the internal and symptom curves. $P$ is the maximal rate of internal elements of the systems. $S$ is the overall sensitivity to the environment. $L$ is the level of predisposing factors. $\lambda_{f}$ is the scaling factor of the slow evolution of fluctuations affecting $L$. $\zeta(t)$ is a point in the normal distribution where $\sigma = 0.5$. It is calculated at each $0.01t$, and is clamped between -1 and 1. 

\paragraph{Parameter settings}
There are four initial parameter settings that we have taken from the same source \citep{gauldDynamicalSystemsComputational2023}. They represent four different disorders. Their initial conditions are given at page~\pageref{tab:1}. Each time series is representative of a different kind of chaotic behaviour. The time series of the `healthy'-trajectory moves randomly around $0.1$. The time series of the `schizophrenia' time series moves close to 8, before dropping for intervals. Both `bereavement' and `bipolar' oscillate quickly in symptom strength, covering the full total range. For visualizations, see page~\pageref{fig:1}. 

\subsubsection{Solvers}
We used the Tsitouras 5/4 Runge-Kutta method as the solver for the differential equations, as implemented in the DifferentialEquations.jl package \citep{tsitourasRungeKuttaPairs2011}. We used standard settings for all of the parameters, aside from a higher number of maximum iterations ($1e^{7}$). The baseline is calculated for $0.01t$, where $t$ represents one day in the model. 

\subsection{Stage 2: Binning data and removing time points}
Afterwards, we systematically reduced the quality of the data. We binned the range of the width of the data into $n$ intervals of equal length, where $n$ stands for the number of bins. Moreover, we removed time points from the data by keeping the first and every $k^{th}$ observation of the simulated data.
We systematically decreased the number of bins and the number of time points, and re-analyze the data. $k$ at 1 is set at the baseline. This implies no reduction. The other $k$-values include 2, 4, and 8. For binning, $n$ = 100 is set at the baseline, and is equivalent to a visual analog scale. Other $n$-values include 20, 7, 6, 5, 4, 3, and 2. These were chosen to reflect different types of measuring instruments, such as several types of likert and forced-choice scales.

\subsection{Stage 3: Data analysis}
We judged the sensitivity of the data by deriving the recurrence indicators introduced before for each time series in each state of degradation. We calculated the deviation of each of these values from the baseline, which are the recurrence values derived for the intact dataset. We mapped the changes as the deviation for these indicators between the baseline and a set of degraded data, adjusting indicators based on line length by multiplying them by the reduction factor.

\subsubsection{Recurrence Quantification Analysis}
RQA is a method that is based on the identification of recurrent points in a time series. A point recurs if it is within the recurrence threshold of another point in time \citep{webber2005recurrence}. Indicators can then be derived from this matrix. The development of these indicators has seen considerable development \citep{marwanTrendsRecurrenceAnalysis2023}. We chose to focus on the core set of indicators, as described by Marwan and Webber \citep{marwanMathematicalComputationalFoundations2015}. The recurrence threshold was set at the size of the bins of the degraded data set. E.g., if the range of the trajectory was 0 to 2, and the number of bins is 7 (data is degraded so that it is similar to likert-scale data), then the recurrence threshold would have been set at $\frac{2-0}{7}$ = $\frac{2}{7}$. A visualization of these recurrences for the four trajectories can be found on page~\pageref{fig:2}.

\subsubsection{Recurrence Indicators}
The \textit{recurrence rate} is the proportion of points in the phase space that reoccur at later times \citep{webber2005recurrence}. Higher recurrence rates indicate that an underlying function is more periodic.\ \textit{Determinism} is the share of recurrent points that are part of diagonal lines, which indicate that the structure might be deterministic. It should be noted that it is a necessary condition, not sufficient by itself, to indicate determinism \citep{marwanHowAvoidPotential2011}.\ \textit{Average and maximum length of diagonal structures} are also given. A longer average length means more predictable dynamics. A longer maximum indicates the longest segment.\ \textit{Entropy of diagonal structures} concerns the Shannon entropy of diagonal line lengths \citep{kraemerRecurrenceThresholdSelection2018}. It is an indicator of the amount of randomness, or information, in the data.\ \textit{Trapping time} is the average length of vertical lines in the plot. It is a measure of how long a system stays in a particular state.\ \textit{Most probable recurrence time}, similarly, is the mode of the length of the vertical lines in the plot. 

\subsection{Software}
We used the Julia language, and in particular the `DynamicalSystems.jl', `RecurrenceAnalysis.jl', and `Statistics.jl' packages to implement the toy model and run the recurrence analyses \citep{bezanson2017julia, Datseris2018, DatserisParlitz2022}. Analyses were run on a personal computer. Full information about dependencies and version numbers can be found in a machine-readable format in the Manifest.toml file in the Github-repository. Instructions for running the analysis through a sandboxed project environment identical to our system can be found on the main page of this repository.

% For Original Research articles, please note that the Material and Methods section can be placed in any of the following ways: before Results, before Discussion or after Discussion.


%%Figures, tables, and images will be published under a Creative Commons CC-BY licence and permission must be obtained for use of copyrighted material from other sources (including re-published/adapted/modified/partial figures and images from the internet). It is the responsibility of the authors to acquire the licenses, to follow any citation instructions requested by third-party rights holders, and cover any supplementary charges.

\section*{Conflict of Interest Statement}
The authors declare that the research was conducted in the absence of any commercial or financial relationships that could be construed as a potential conflict of interest.

\section*{Funding}
No external funding was used for this project.

\section*{Acknowledgments}
I acknowledge the work of my thesis supervisors, who introduced me to the method and left me free to pursue the project as I imagined it. The great help of the Julia community was also appreciated, as they have been pushing me forward where I got stuck and took the time to respond to my stupid questions. Finally, I would like to acknowledge the feedback and conversations between me and my thesis group, who have been working through my text and made sure that it is easy to follow and well-written. Special thanks to Giuliana Orizzonte, Daniel Anadria, dr. Derksen, and dr. Bringmann for fruitful discussions and feedback about my topic. Lastly, I would like to thank my girlfriend, family, and friends for the mental support throughout.

\section*{Data Availability Statement}
The code, additional material, and generated data for this study can be found on \href{https://github.com/MvanSteenbergen/MasterThesisRQA}{GitHub}.

% Please see the availability of data guidelines for more information, at https://www.frontiersin.org/about/author-guidelines#AvailabilityofData

\bibliographystyle{Frontiers-Harvard} %  Many Frontiers journals use the Harvard referencing system (Author-date), to find the style and resources for the journal you are submitting to: https://zendesk.frontiersin.org/hc/en-us/articles/360017860337-Frontiers-Reference-Styles-by-Journal. For Humanities and Social Sciences articles please include page numbers in the in-text citations 
%\bibliographystyle{Frontiers-Vancouver} % Many Frontiers journals use the numbered referencing system, to find the style and resources for the journal you are submitting to: https://zendesk.frontiersin.org/hc/en-us/articles/360017860337-Frontiers-Reference-Styles-by-Journal
\newpage
\bibliography{bibliography}
\newpage
\section*{Figures}
\begin{figure}[H]
    \begin{center}
    \includegraphics[width=15cm]{time_series_plot}
    \end{center}
    \caption{A section of the time series created using the coupled differential equations and parameter settings specified in section 2.1.1. This is the intact data, before degradation takes place.}\label{fig:1}
    \end{figure}

\newpage
\begin{figure}[H]
    \begin{center}
    \includegraphics[width=15cm]{recurrence_plots}
    \end{center}
    \caption{Recurrence plot for the four time series generated using the coupled differential equations and parameter settings specified in section 2.1.1. A point recurs when it is within the recurrence threshold of another point. Recurrent points are black, non-recurrent points are white. The axes represent time points, each location on the matrix represents a combination of time points. The recurrence threshold is set at 0.2 for illustration purpose. Note that the plot for the `healthy' trajectory is completely black: this is because every point in the plot falls within the recurrence threshold. Also note the black `boxes' where the bottom two trajectories are stagnant.}\label{fig:2}
    \end{figure}

\newpage
\section*{Tables}
\begin{table}[H]
    \begin{tabular}{|l| c c c c c c c c c c c c c c c|}
    \hline
    \bf{Parameter} & $S_{max}$ & $R_{s}$ & $\lambda_{s}$ & $\tau_{x}$ & $P$ & $R_{b}$ & $\lambda_{b}$ & $L$ & $\tau_{y}$ & $S$ & $\alpha$ & $\beta$ & $\tau_{z}$ & $\lambda_{d}$ & $\tau_{f}$ \\
    \hline
    \it{Healthy} & 10 & 1 & 0.1 & 14 & 10 & 1.04 & 0.05 & 0.2 & 14 & 4 & 0.5 & 0.5 & 1 & 1 & 720 \\
    \hline
    \it{Schizophrenia} & 10 & 1 & 0.1 & 14 & 10 & 0.904 & 0.05 & 0.2 & 14 & 4 & 0.5 & 0.5 & 1 & 1 & 720 \\
    \hline
    \it{Bipolar} & 10 & 1 & 0.1 & 14 & 10 & 1.04 & 0.05 & 1.01 & 14 & 10 & 0.5 & 0.5 & 1 & 1 & 720 \\
    \hline
    \it{Bereavement} & 10 & 1 & 0.1 & 14 & 10 & 1 & 0.05 & 0.6 & 14 & 4.5 & 0.5 & 0.5 & 1 & 1 & 720 \\
    \hline
    \end{tabular}
    \caption{The parameter settings used as initial parameter settings for the coupled differential equations specified in paragraph 2.1.1}\label{tab:1}
    \end{table}



    
%%% Make sure to upload the bib file along with the tex file and PDF
%%% Please see the test.bib file for some examples of references

%%% Please be aware that for original research articles we only permit a combined number of 15 figures and tables, one figure with multiple subfigures will count as only one figure.
%%% Use this if adding the figures directly in the mansucript, if so, please remember to also upload the files when submitting your article
%%% There is no need for adding the file termination, as long as you indicate where the file is saved. In the examples below the files (logo1.eps and logos.eps) are in the Frontiers LaTeX folder
%%% If using *.tif files convert them to .jpg or .png
%%%  NB logo1.eps is required in the path in order to correctly compile front page header %%%


%%% If you don't add the figures in the LaTeX files, please upload them when submitting the article.
%%% Frontiers will add the figures at the end of the provisional pdf automatically
%%% The use of LaTeX coding to draw Diagrams/Figures/Structures should be avoided. They should be external callouts including graphics.

\end{document}