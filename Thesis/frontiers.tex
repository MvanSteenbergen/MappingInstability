%%%%%%%%%%%%%%%%%%%%%%%%%%%%%%%%%%%%%%%%%%%%%%%%%%%%%%%%%%%%%%%%%%%%%%%%%%%%%%%%%%%%%%%%%%%%%%%%%%%%%%%%%%%%%%%%%%%%%%%%%%%%%%%%%%%%%%%%%%%%%%%%%%%%%%%%%%%
% This is just an example/guide for you to refer to when submitting manuscripts to Frontiers, it is not mandatory to use Frontiers .cls files nor frontiers.tex  %
% This will only generate the Manuscript, the final article will be typeset by Frontiers after acceptance.   
%                                              %
%                                                                                                                                                         %
% When submitting your files, remember to upload this *tex file, the pdf generated with it, the *bib file (if bibliography is not within the *tex) and all the figures.
%%%%%%%%%%%%%%%%%%%%%%%%%%%%%%%%%%%%%%%%%%%%%%%%%%%%%%%%%%%%%%%%%%%%%%%%%%%%%%%%%%%%%%%%%%%%%%%%%%%%%%%%%%%%%%%%%%%%%%%%%%%%%%%%%%%%%%%%%%%%%%%%%%%%%%%%%%%

%%% Version 3.4 Generated 2022/06/14 %%%
%%% You will need to have the following packages installed: datetime, fmtcount, etoolbox, fcprefix, which are normally inlcuded in WinEdt. %%%
%%% In http://www.ctan.org/ you can find the packages and how to install them, if necessary. %%%
%%%  NB logo1.jpg is required in the path in order to correctly compile front page header %%%

\documentclass[utf8]{FrontiersinVancouver} 
\usepackage{url,hyperref,lineno,microtype,subcaption}
\usepackage[onehalfspacing]{setspace}

\linenumbers


% Leave a blank line between paragraphs instead of using \\


\def\keyFont{\fontsize{8}{11}\helveticabold }
\def\firstAuthorLast{Sample {et~al.}} %use et al only if is more than 1 author
\def\Authors{Maas van Steenbergen\,$^{1,*}$}
% Affiliations should be keyed to the author's name with superscript numbers and be listed as follows: Laboratory, Institute, Department, Organization, City, State abbreviation (USA, Canada, Australia), and Country (without detailed address information such as city zip codes or street names).
% If one of the authors has a change of address, list the new address below the correspondence details using a superscript symbol and use the same symbol to indicate the author in the author list.
\def\Address{$^{1}$Laboratory X, Faculty of Behavioural and Social Sciences,  Methodology \& Statistics, Utrecht University, the Netherlands  }
% The Corresponding Author should be marked with an asterisk
% Provide the exact contact address (this time including street name and city zip code) and email of the corresponding author
\def\corrAuthor{Corresponding Author}

\def\corrEmail{m.vansteenbergen@uu.nl}


\begin{document}
\onecolumn
\firstpage{1}

\title[Running Title]{Making Self-Report Ready for Dynamics: the Impact of Low Sampling Frequency and Bandwidth on Recurrence Quantification Analysis in Idiographic Ecological Momentary Assessment} 

\author[\firstAuthorLast ]{\Authors} %This field will be automatically populated
\address{} %This field will be automatically populated
\correspondance{} %This field will be automatically populated

\extraAuth{}% If there are more than 1 corresponding author, comment this line and uncomment the next one.
%\extraAuth{corresponding Author2 \\ Laboratory X2, Institute X2, Department X2, Organization X2, Street X2, City X2 , State XX2 (only USA, Canada and Australia), Zip Code2, X2 Country X2, email2@uni2.edu}


\maketitle

\begin{abstract}

%%% Leave the Abstract empty if your article does not require one, please see the Summary Table for full details.
\section{}
For full guidelines regarding your manuscript please refer to \href{http://www.frontiersin.org/about/AuthorGuidelines}{Author Guidelines}.

As a primary goal, the abstract should render the general significance and conceptual advance of the work clearly accessible to a broad readership. References should not be cited in the abstract. Leave the Abstract empty if your article does not require one, please see \href{http://www.frontiersin.org/about/AuthorGuidelines#SummaryTable}{Summary Table} for details according to article type. 


\tiny
 \keyFont{ \section{Keywords:} complex dynamics, data quality} %All article types: you may provide up to 8 keywords; at least 5 are mandatory.
\end{abstract}

\section{Introduction}
Self-report scales have a long historical precedent in psychology. Ecological momentary assessment (EMA) is a technique meant to construct time series based on self-report instruments. This approach allows mapping within-person fluctuations of psychological constructs in a systematic manner \citep{connerExperienceSamplingMethods2009}. While traditional statistical methods are frequently and fruitfully employed to analyze data generated using EMA, these methods are not suitable for capturing complex temporal within-person patterns \citep{olthofPsychologicalDynamicsAre2020}. 

We postulate that there is an underlying trajectory of whatever we aim to measure. These  In other words, we assume that the categorical We also assume that the trajectory of the time series is chaotic, which means that the data generating mechanism is very sensitive to changes in the initial parameters. This means that the data generated using these methods is only predictable in the short-term, and that observations are dependent on the state of the system and its externalities at earlier time points \citep{olthofComplexityPsychologicalSelfratings2020b}. Hence, chaotic behaviour can be easily mistaken for random behaviour, but can (yet does not have to be) fully deterministic. We also postulate that self-report measures are accurate measures at a certain timepoint of this continuous measure, but with lower bandwith and more coursegraining. Working from these assumptions, we run a simulation 

The methods to study chaotic behavior are still in infancy within a psychological context \citep{molenaarManifestoPsychologyIdiographic2004}. They are often imported from complex dynamical systems theory, which is an area of mathematics that concerns itself with the study of the time-dependent dynamics of complex systems. A popular analysis technique is called Recurrence Quantification Analysis (RQA). It results in the identification of recurrent patterns, or repetitions, in time series analysis \citep{webber2005recurrence}. One can then derive several indicators of the stability, predictability, and dynamical behavior of data from these recurrences. This method was developed in the physical sciences under the assumption that measurements can be retrieved at great frequency and at high resolution, to an extent that is impossible when relying on self-report scales. Hence, it is necessary to systematically assess the consequences of utilizing EMA data on the quality of RQA output \citep{haslbeckRecoveringWithinpersonDynamics2022}.

We choose to generate the data through a set of coupled differential equations that lead to chaotic behaviour. Note that randomness has a part to play in the generation of the data, but that most of the variance is created determinalistically through four broadly modeled influences. These influences include symptom intensity, the modelling of the person's internal factors, influences of the perceived environment, and the influence of time. Many alternative models could be formulated that could result in similar behavior or would be as realistic. 

Our choice for this way of modelling is based on a variety of interesting characteristics. The most important of these is that self-report data cannot be measured consistently and continuously with high ecological validity in almost all use cases where self-report is used. We note that if it were possible to measure these variables directly, then researchers would not have to rely on the instrument in the first place. Therefore, theory-based simulation is a necessary starting point for further research. We hope that the models, in turn, can be refined using empirical data that take into account the time dependence, and that is where the recovery of complex characteristics can be used in ways that would be hard to do if one were to study the trajectory statistically.

This brings us to a natural place to introduce the goal of this project: we hope that studying the stability of complexity characteristics under data degredation can help us recover certain aspects of the trajectory of the system under study. E.g., incompatibility of the theoretical time development of disease symptomatology contrasted with the recovered complexity characteristics of a patient can be used as a counterfactual against the choice of one model over another. A more concrete example: say that a patient is postulated to have a disorder where a particular symptom oscillates predictably and repeatadly for a period of 12 hours. A more fine-tuned understanding of that trajectory can then be used to optimize treatment times. If their symptoms were to oscillate more irratically, then this would be reflected in the recurrence indicators of the patient. If it would show dampened oscillations instead where the bandwith becomes lower with the passage of time, then that would be reflected in the recurrence inidicators in a different manner. This is, of course, only true if those characteristics can be picked up reliably from the degraded data. 

\section{Materials and methods}

\subsection{Software}
We used the Julia languages, and in particular `DynamicalSystems.jl', `RecurrenceAnalysis.jl', and `Statistics.jl', to implement the toy model and run the recurrence analyses. Full information about dependencies and version numbers can be found in a human-readable format in the Manifest.toml file in the Github-repository. Instructions for running the analysis through a sandboxed project environment that is identical can be found on the main page of the github repository.

\subsection{Toy Model}
For this study, we will use the "3 + 1 dimension model" introduced by \citep{gauldDynamicalSystemsComputational2023}. The original aim of this toy model is to simulate the trajectory of symptomatology over time, but it can be used for our project by reformulating some of the model .  It uses four coupled differential equations to model the effect of time on symptom intensity. It is explained quite well in the aforementioned paper. We give a basic explanation of each equation, and note some interesting behaviour that might be more or less suitable for use in this project. I will use identical terminology where possible


\subsubsection{Symptom intensity}
The first equation is supposed to represent symptom intensity. 

\begin{equation}
    \tau_{x}\frac{dx}{dt} = \frac{S_{\max}}{1+exp(\frac{Rs-y}{\lambda_{s}})} - x
\end{equation}

\subsubsection{Modelling of internal elements}

\begin{equation}
    \tau_{y}\frac{dy}{dt} = \frac{P}{1+exp(\frac{R_{b}-y}{\lambda_{b}})} + L - xy - z
\end{equation}

\subsubsection{Modelling of perceived environment}

\begin{equation}
    \tau_{y}\frac{dy}{dt} = \frac{P}{1+exp(\frac{R_{b}-y}{\lambda_{b}})} + L - xy - z
\end{equation}

\subsubsection{Temporal specificities}

\begin{equation}
    \tau_{z}\frac{dz}{dt} = S(ax + \beta y)\zeta(t) - z
\end{equation}

\subsection{Recurrence Indicators}

\textbf{Recurrence rate}
\textbf{Determinism}
\textbf{Average length of diagonal structures}
\textbf{Maximum length of diagonal structures}
\textbf{Divergence}
\textbf{Entropy of diagonal structures}
\text
\textbf{Trapping time}
\textbf{Most probable recurrence time}

\section{Results}

\section{Discussion}
The results of this study suggest that applying recurrence methods in 

One of the ways that researchers can strengthen their idiographic inferences is through asking whether their 

The limitations of this study are 
\section{Article types}

For requirements for a specific article type please refer to the Article Types on any Frontiers journal page. Please also refer to  \href{http://home.frontiersin.org/about/author-guidelines#Sections}{Author Guidelines} for further information on how to organize your manuscript in the required sections or their equivalents for your field

% For Original Research articles, please note that the Material and Methods section can be placed in any of the following ways: before Results, before Discussion or after Discussion.

\section{Manuscript Formatting}

\subsection{Heading Levels}

%There are 5 heading levels

\subsection{Level 2}
\subsubsection{Level 3}
\paragraph{Level 4}
\subparagraph{Level 5}

\subsection{Equations}
Equations should be inserted in editable format from the equation editor.

\begin{equation}
\sum x+ y =Z\label{eq:01}
\end{equation}

\subsection{Figures}
Frontiers requires figures to be submitted individually, in the same order as they are referred to in the manuscript. Figures will then be automatically embedded at the bottom of the submitted manuscript. Kindly ensure that each table and figure is mentioned in the text and in numerical order. Figures must be of sufficient resolution for publication \href{https://www.frontiersin.org/about/author-guidelines#ImageSizeRequirements}{see here for examples and minimum requirements}. Figures which are not according to the guidelines will cause substantial delay during the production process. Please see \href{https://www.frontiersin.org/about/author-guidelines#FigureRequirementsStyleGuidelines}{here} for full figure guidelines. Cite figures with subfigures as figure \ref{fig:Subfigure 1} and \ref{fig:Subfigure 2}.


\subsubsection{Permission to Reuse and Copyright}
Figures, tables, and images will be published under a Creative Commons CC-BY licence and permission must be obtained for use of copyrighted material from other sources (including re-published/adapted/modified/partial figures and images from the internet). It is the responsibility of the authors to acquire the licenses, to follow any citation instructions requested by third-party rights holders, and cover any supplementary charges.
%%Figures, tables, and images will be published under a Creative Commons CC-BY licence and permission must be obtained for use of copyrighted material from other sources (including re-published/adapted/modified/partial figures and images from the internet). It is the responsibility of the authors to acquire the licenses, to follow any citation instructions requested by third-party rights holders, and cover any supplementary charges.

\subsection{Tables}
Tables should be inserted at the end of the manuscript. Please build your table directly in LaTeX.Tables provided as jpeg/tiff files will not be accepted. Please note that very large tables (covering several pages) cannot be included in the final PDF for reasons of space. These tables will be published as \href{http://home.frontiersin.org/about/author-guidelines#SupplementaryMaterial}{Supplementary Material} on the online article page at the time of acceptance. The author will be notified during the typesetting of the final article if this is the case. 

\subsection{International Phonetic Alphabet}
To include international phonetic alphabet (IPA) symbols, please include the following functions:
Under useful packages, include:\begin{verbatim}\usepackage{tipa}\end{verbatim} 
In the main text, when inputting symbols, use the following format:\begin{verbatim}\text[symbolname]\end{verbatim}e.g.\begin{verbatim}\textgamma\end{verbatim}

\section{Nomenclature}

\subsection{Resource Identification Initiative}
To take part in the Resource Identification Initiative, please use the corresponding catalog number and RRID in your current manuscript. For more information about the project and for steps on how to search for an RRID, please click \href{http://www.frontiersin.org/files/pdf/letter_to_author.pdf}{here}.



\section{Additional Requirements}

For additional requirements for specific article types and further information please refer to \href{http://www.frontiersin.org/about/AuthorGuidelines#AdditionalRequirements}{Author Guidelines}.

\section*{Conflict of Interest Statement}
The authors declare that the research was conducted in the absence of any commercial or financial relationships that could be construed as a potential conflict of interest.

\section*{Author Contributions}

The Author Contributions section is mandatory for all articles, including articles by sole authors. If an appropriate statement is not provided on submission, a standard one will be inserted during the production process. The Author Contributions statement must describe the contributions of individual authors referred to by their initials and, in doing so, all authors agree to be accountable for the content of the work. Please see  \href{https://www.frontiersin.org/about/policies-and-publication-ethics#AuthorshipAuthorResponsibilities}{here} for full authorship criteria.

\section*{Funding}
Details of all funding sources should be provided, including grant numbers if applicable. Please ensure to add all necessary funding information, as after publication this is no longer possible.

\section*{Acknowledgments}
I acknowledge the work of my thesis supervisors, who introduced me to the method and left me free to persue the project as I imagined it. I also acknowledge the great help of the Julia community, which has been helping me with programming where I got stuck and which took the time to respond to my stupid questions. Finally, I'd like to acknowledge the feedback and conversations between me and my thesis group, who have been working through my text and made sure that it is followable.

\section*{Supplemental Data}
 \href{http://home.frontiersin.org/about/author-guidelines#SupplementaryMaterial}{Supplementary Material} should be uploaded separately on submission, if there are Supplementary Figures, please include the caption in the same file as the figure. LaTeX Supplementary Material templates can be found in the Frontiers LaTeX folder.

\section*{Data Availability Statement}
The code, all additional material, and generated data for this study can be found in the [NAME OF REPOSITORY]{}.
% Please see the availability of data guidelines for more information, at https://www.frontiersin.org/about/author-guidelines#AvailabilityofData

\bibliographystyle{Frontiers-Harvard} %  Many Frontiers journals use the Harvard referencing system (Author-date), to find the style and resources for the journal you are submitting to: https://zendesk.frontiersin.org/hc/en-us/articles/360017860337-Frontiers-Reference-Styles-by-Journal. For Humanities and Social Sciences articles please include page numbers in the in-text citations 
%\bibliographystyle{Frontiers-Vancouver} % Many Frontiers journals use the numbered referencing system, to find the style and resources for the journal you are submitting to: https://zendesk.frontiersin.org/hc/en-us/articles/360017860337-Frontiers-Reference-Styles-by-Journal
\bibliography{bibliography}

%%% Make sure to upload the bib file along with the tex file and PDF
%%% Please see the test.bib file for some examples of references

\section*{Figure captions}

%%% Please be aware that for original research articles we only permit a combined number of 15 figures and tables, one figure with multiple subfigures will count as only one figure.
%%% Use this if adding the figures directly in the mansucript, if so, please remember to also upload the files when submitting your article
%%% There is no need for adding the file termination, as long as you indicate where the file is saved. In the examples below the files (logo1.eps and logos.eps) are in the Frontiers LaTeX folder
%%% If using *.tif files convert them to .jpg or .png
%%%  NB logo1.eps is required in the path in order to correctly compile front page header %%%

\begin{figure}[h!]
\begin{center}
\includegraphics[width=10cm]{logo1}% This is a *.eps file
\end{center}
\caption{ Enter the caption for your figure here.  Repeat as  necessary for each of your figures}\label{fig:1}
\end{figure}

\setcounter{figure}{2}
\setcounter{subfigure}{0}
\begin{subfigure}
\setcounter{figure}{2}
\setcounter{subfigure}{0}
    \centering
    \begin{minipage}[b]{0.5\textwidth}
        \includegraphics[width=\linewidth]{logo1.eps}
        \caption{This is Subfigure 1.}
        \label{fig:Subfigure 1}
    \end{minipage}  
   
\setcounter{figure}{2}
\setcounter{subfigure}{1}
    \begin{minipage}[b]{0.5\textwidth}
        \includegraphics[width=\linewidth]{logo2.eps}
        \caption{This is Subfigure 2.}
        \label{fig:Subfigure 2}
    \end{minipage}

\setcounter{figure}{2}
\setcounter{subfigure}{-1}
    \caption{Enter the caption for your subfigure here. \textbf{(A)} This is the caption for Subfigure 1. \textbf{(B)} This is the caption for Subfigure 2.}
    \label{fig: subfigures}
\end{subfigure}

%%% If you don't add the figures in the LaTeX files, please upload them when submitting the article.
%%% Frontiers will add the figures at the end of the provisional pdf automatically
%%% The use of LaTeX coding to draw Diagrams/Figures/Structures should be avoided. They should be external callouts including graphics.

\end{document}
