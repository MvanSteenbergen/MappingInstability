% Options for packages loaded elsewhere
\PassOptionsToPackage{unicode}{hyperref}
\PassOptionsToPackage{hyphens}{url}
\PassOptionsToPackage{dvipsnames,svgnames,x11names}{xcolor}
%
\documentclass[
  letterpaper,
  DIV=11,
  numbers=noendperiod]{scrartcl}

\usepackage{amsmath,amssymb}
\usepackage{iftex}
\ifPDFTeX
  \usepackage[T1]{fontenc}
  \usepackage[utf8]{inputenc}
  \usepackage{textcomp} % provide euro and other symbols
\else % if luatex or xetex
  \usepackage{unicode-math}
  \defaultfontfeatures{Scale=MatchLowercase}
  \defaultfontfeatures[\rmfamily]{Ligatures=TeX,Scale=1}
\fi
\usepackage{lmodern}
\ifPDFTeX\else  
    % xetex/luatex font selection
\fi
% Use upquote if available, for straight quotes in verbatim environments
\IfFileExists{upquote.sty}{\usepackage{upquote}}{}
\IfFileExists{microtype.sty}{% use microtype if available
  \usepackage[]{microtype}
  \UseMicrotypeSet[protrusion]{basicmath} % disable protrusion for tt fonts
}{}
\makeatletter
\@ifundefined{KOMAClassName}{% if non-KOMA class
  \IfFileExists{parskip.sty}{%
    \usepackage{parskip}
  }{% else
    \setlength{\parindent}{0pt}
    \setlength{\parskip}{6pt plus 2pt minus 1pt}}
}{% if KOMA class
  \KOMAoptions{parskip=half}}
\makeatother
\usepackage{xcolor}
\setlength{\emergencystretch}{3em} % prevent overfull lines
\setcounter{secnumdepth}{-\maxdimen} % remove section numbering
% Make \paragraph and \subparagraph free-standing
\ifx\paragraph\undefined\else
  \let\oldparagraph\paragraph
  \renewcommand{\paragraph}[1]{\oldparagraph{#1}\mbox{}}
\fi
\ifx\subparagraph\undefined\else
  \let\oldsubparagraph\subparagraph
  \renewcommand{\subparagraph}[1]{\oldsubparagraph{#1}\mbox{}}
\fi


\providecommand{\tightlist}{%
  \setlength{\itemsep}{0pt}\setlength{\parskip}{0pt}}\usepackage{longtable,booktabs,array}
\usepackage{calc} % for calculating minipage widths
% Correct order of tables after \paragraph or \subparagraph
\usepackage{etoolbox}
\makeatletter
\patchcmd\longtable{\par}{\if@noskipsec\mbox{}\fi\par}{}{}
\makeatother
% Allow footnotes in longtable head/foot
\IfFileExists{footnotehyper.sty}{\usepackage{footnotehyper}}{\usepackage{footnote}}
\makesavenoteenv{longtable}
\usepackage{graphicx}
\makeatletter
\def\maxwidth{\ifdim\Gin@nat@width>\linewidth\linewidth\else\Gin@nat@width\fi}
\def\maxheight{\ifdim\Gin@nat@height>\textheight\textheight\else\Gin@nat@height\fi}
\makeatother
% Scale images if necessary, so that they will not overflow the page
% margins by default, and it is still possible to overwrite the defaults
% using explicit options in \includegraphics[width, height, ...]{}
\setkeys{Gin}{width=\maxwidth,height=\maxheight,keepaspectratio}
% Set default figure placement to htbp
\makeatletter
\def\fps@figure{htbp}
\makeatother
\newlength{\cslhangindent}
\setlength{\cslhangindent}{1.5em}
\newlength{\csllabelwidth}
\setlength{\csllabelwidth}{3em}
\newlength{\cslentryspacingunit} % times entry-spacing
\setlength{\cslentryspacingunit}{\parskip}
\newenvironment{CSLReferences}[2] % #1 hanging-ident, #2 entry spacing
 {% don't indent paragraphs
  \setlength{\parindent}{0pt}
  % turn on hanging indent if param 1 is 1
  \ifodd #1
  \let\oldpar\par
  \def\par{\hangindent=\cslhangindent\oldpar}
  \fi
  % set entry spacing
  \setlength{\parskip}{#2\cslentryspacingunit}
 }%
 {}
\usepackage{calc}
\newcommand{\CSLBlock}[1]{#1\hfill\break}
\newcommand{\CSLLeftMargin}[1]{\parbox[t]{\csllabelwidth}{#1}}
\newcommand{\CSLRightInline}[1]{\parbox[t]{\linewidth - \csllabelwidth}{#1}\break}
\newcommand{\CSLIndent}[1]{\hspace{\cslhangindent}#1}

\KOMAoption{captions}{tableheading}
\makeatletter
\makeatother
\makeatletter
\makeatother
\makeatletter
\@ifpackageloaded{caption}{}{\usepackage{caption}}
\AtBeginDocument{%
\ifdefined\contentsname
  \renewcommand*\contentsname{Table of contents}
\else
  \newcommand\contentsname{Table of contents}
\fi
\ifdefined\listfigurename
  \renewcommand*\listfigurename{List of Figures}
\else
  \newcommand\listfigurename{List of Figures}
\fi
\ifdefined\listtablename
  \renewcommand*\listtablename{List of Tables}
\else
  \newcommand\listtablename{List of Tables}
\fi
\ifdefined\figurename
  \renewcommand*\figurename{Figure}
\else
  \newcommand\figurename{Figure}
\fi
\ifdefined\tablename
  \renewcommand*\tablename{Table}
\else
  \newcommand\tablename{Table}
\fi
}
\@ifpackageloaded{float}{}{\usepackage{float}}
\floatstyle{ruled}
\@ifundefined{c@chapter}{\newfloat{codelisting}{h}{lop}}{\newfloat{codelisting}{h}{lop}[chapter]}
\floatname{codelisting}{Listing}
\newcommand*\listoflistings{\listof{codelisting}{List of Listings}}
\makeatother
\makeatletter
\@ifpackageloaded{caption}{}{\usepackage{caption}}
\@ifpackageloaded{subcaption}{}{\usepackage{subcaption}}
\makeatother
\makeatletter
\@ifpackageloaded{tcolorbox}{}{\usepackage[skins,breakable]{tcolorbox}}
\makeatother
\makeatletter
\@ifundefined{shadecolor}{\definecolor{shadecolor}{rgb}{.97, .97, .97}}
\makeatother
\makeatletter
\makeatother
\makeatletter
\makeatother
\ifLuaTeX
  \usepackage{selnolig}  % disable illegal ligatures
\fi
\IfFileExists{bookmark.sty}{\usepackage{bookmark}}{\usepackage{hyperref}}
\IfFileExists{xurl.sty}{\usepackage{xurl}}{} % add URL line breaks if available
\urlstyle{same} % disable monospaced font for URLs
\hypersetup{
  pdftitle={Introduction},
  pdfauthor={Maas van Steenbergen},
  colorlinks=true,
  linkcolor={blue},
  filecolor={Maroon},
  citecolor={Blue},
  urlcolor={Blue},
  pdfcreator={LaTeX via pandoc}}

\title{Introduction}
\usepackage{etoolbox}
\makeatletter
\providecommand{\subtitle}[1]{% add subtitle to \maketitle
  \apptocmd{\@title}{\par {\large #1 \par}}{}{}
}
\makeatother
\subtitle{The effect of low sampling frequency and bandwith on
Recurrence Quantification Analysis for Ecological Mometary Assessment}
\author{Maas van Steenbergen}
\date{2023-10-03}

\begin{document}
\maketitle
\ifdefined\Shaded\renewenvironment{Shaded}{\begin{tcolorbox}[sharp corners, boxrule=0pt, enhanced, borderline west={3pt}{0pt}{shadecolor}, frame hidden, interior hidden, breakable]}{\end{tcolorbox}}\fi

Ecological momentary assessment has made it possible to construct time
series on the basis of self-report scales, where one can map
idiographic, within-person fluctuations of psychological constructs in a
systematic manner (Conner et al. 2009). Data collected using these
methods have been shown to display all markers of complex dynamics,
which means that the future of the data generated using these methods is
only predictable in the short-term, and that observations are dependent
on the state of the system and its externalities at earlier timepoints
(Olthof, Hasselman, and Lichtwarck-Aschoff 2020). Traditional
statistical methods are often used to analyze the data that is generated
using EMA, but these methods are not suitable to capture complex
temporal patterns of psychological constructs within one person(Jenkins
et al. 2020).

Recurrence quantification analysis is an analysis technique that can be
used to aid in understanding psychological dynamics. This method
identifes recurrent patterns, or repetitions, in a time series (Webber
Jr and Zbilut 2005). The method results in several indicators that can
be used to understand the stability, predictability, and dynamical
behavior of the construct. While these methods were developed under the
assumption that measurements can be retrieved at great frequency and at
high resolution, this is generally not the case for EMA. It relies on
the admission of tests taken several times a day, meaning that the
sampling frequency is limited (Haslbeck and Ryan 2022).Moreover, the
psychological constructs that are measured using EMA cannot be measured
without relying on ordinal self-report questionnaires. Therefore,
systematic study of the consequences of these limitations on RQA is
needed.

For the purposes of this project, we assume that the underlying
psychological construct is a continuously changing dynamical value
(Boker 2002), and that EMA output values are accurate ordinal,
relatively low-t attempts to measure continuous underlying dynamical
processes. This is an idealized assumption to study the consequences of
low sampling frequency and data bandwith, and does not take into account
possible challenges to ecological validity (Stinson, Liu, and Dallery
2022). This is outside of the scope of this project .

\hypertarget{the-current-project}{%
\section{The current project}\label{the-current-project}}

This project aims to find out at what point decreased data quality
limits the ability of EMA to capture idiographic dynamics. We present an
analysis pipeline consisting of multiple stages. We will use the
\texttt{DynamicalSystems.jl} and \texttt{Statistics.jl} julia-packages
to simulate the toy model and perform the analysis (Bezanson et al.
2017; Datseris 2018; Datseris and Parlitz 2022).

\hypertarget{stage-1-data-generation}{%
\subsubsection{Stage 1: Data generation}\label{stage-1-data-generation}}

In the first stage, we use a toy model developed to simulate the data
based on a 3 + 1 dimensions model (Gauld and Depannemaecker 2023). This
model captures clinical observations found in psychiatric symptomology
by modeling internal factors (\(y\)), environmental noise (\(z\)),
temporal specificities (\(f\)), and sympomatology (\(x\)).
Symptomatology will be the outcome variable of these study. By changing
all four of these coefficients systematically, we aim to model a large
variety of possible trajectories, and we save each one of these models
as a separate time series. For the purpose of our study, we redefine
``symptomatology'' as any dynamical fluctuations of psychological
constructs. It is important to note that this set of equations is not
chosen to be exhaustive, but because it is one of the only systematic
attempts to explicitly model the temporal fluctuations of psychological
constructs.

\hypertarget{stage-2-binning-data-and-removing-time-points}{%
\subsubsection{Stage 2: Binning data and removing time
points}\label{stage-2-binning-data-and-removing-time-points}}

Now, we aim to systematically reduce the quality of the data. We bin a
range of the width of the data into \(n\) intervals of equal length,
where \(n\) stands for the number of bins. We also vary the minimum
(\(min\)) and maximum (\(max\)) value of this range to simulate ceiling
and floor-effects. Moreover, we remove time points from the data by
keeping every \(k\)\textsuperscript{th} observation of the simulated
data. We systematically decrease the value of \(n\) and \(min\) and
increase the value of \(k\) and \(max\), storing any combination of
these values.

\hypertarget{stage-3-data-analysis}{%
\subsubsection{Stage 3: Data analysis}\label{stage-3-data-analysis}}

We will judge the sensitivity of the data by calculating summary
statistics and recurrence indicators (recurrence rate, determinism,
entropy of the distribution.) for each time series in each state of
degradation. We judge the sensitivity of the data to degredation by
looking at the change in values for each of the indicators, where the
full dataset is used as the baseline. We will then map the changes in
the indicators as the difference for that indicator between the baseline
and at \(n\), \(min\), \(max\), and \(k\).

\hypertarget{ethical-approval-and-proof-of-concept}{%
\subsection{Ethical approval and proof of
concept}\label{ethical-approval-and-proof-of-concept}}

The project has been approved by the ethical committee {[}!not yet
done{]}.

\hypertarget{references}{%
\section{References}\label{references}}

\hypertarget{refs}{}
\begin{CSLReferences}{1}{0}
\leavevmode\vadjust pre{\hypertarget{ref-bezanson2017julia}{}}%
Bezanson, Jeff, Alan Edelman, Stefan Karpinski, and Viral B Shah. 2017.
{``Julia: {A} Fresh Approach to Numerical Computing.''} \emph{SIAM
Review} 59 (1): 65--98.

\leavevmode\vadjust pre{\hypertarget{ref-bokerConsequencesContinuityHunt2002}{}}%
Boker, Steven M. 2002. {``Consequences of {Continuity}: {The Hunt} for
{Intrinsic Properties} Within {Parameters} of {Dynamics} in
{Psychological Processes}.''} \emph{Multivariate Behavioral Research} 37
(3): 405--22. \url{https://doi.org/10.1207/S15327906MBR3703_5}.

\leavevmode\vadjust pre{\hypertarget{ref-connerExperienceSamplingMethods2009}{}}%
Conner, Tamlin S., Howard Tennen, William Fleeson, and Lisa Feldman
Barrett. 2009. {``Experience {Sampling Methods}: {A Modern Idiographic
Approach} to {Personality Research}.''} \emph{Social and Personality
Psychology Compass} 3 (3): 292--313.
\url{https://doi.org/10.1111/j.1751-9004.2009.00170.x}.

\leavevmode\vadjust pre{\hypertarget{ref-Datseris2018}{}}%
Datseris, George. 2018. {``{DynamicalSystems}.jl: {A Julia} Software
Library for Chaos and Nonlinear Dynamics.''} \emph{Journal of Open
Source Software} 3 (23): 598. \url{https://doi.org/10.21105/joss.00598}.

\leavevmode\vadjust pre{\hypertarget{ref-DatserisParlitz2022}{}}%
Datseris, George, and Ulrich Parlitz. 2022. \emph{Nonlinear Dynamics:
{A} Concise Introduction Interlaced with Code}. {Cham, Switzerland}:
{Springer Nature}. \url{https://doi.org/10.1007/978-3-030-91032-7}.

\leavevmode\vadjust pre{\hypertarget{ref-gauldDynamicalSystemsComputational2023}{}}%
Gauld, Christophe, and Damien Depannemaecker. 2023. {``Dynamical Systems
in Computational Psychiatry: {A} Toy-Model to Apprehend the Dynamics of
Psychiatric Symptoms.''} \emph{Frontiers in Psychology} 14.

\leavevmode\vadjust pre{\hypertarget{ref-haslbeckRecoveringWithinPersonDynamics2022}{}}%
Haslbeck, Jonas M. B., and Oisín Ryan. 2022. {``Recovering
{Within-Person Dynamics} from {Psychological Time Series}.''}
\emph{Multivariate Behavioral Research} 57 (5): 735--66.
\url{https://doi.org/10.1080/00273171.2021.1896353}.

\leavevmode\vadjust pre{\hypertarget{ref-jenkinsAffectVariabilityPredictability2020}{}}%
Jenkins, Brooke N., John F. Hunter, Michael J. Richardson, Tamlin S.
Conner, and Sarah D. Pressman. 2020. {``Affect Variability and
Predictability: {Using} Recurrence Quantification Analysis to Better
Understand How the Dynamics of Affect Relate to Health.''}
\emph{Emotion} 20 (3): 391--402.
\url{https://doi.org/10.1037/emo0000556}.

\leavevmode\vadjust pre{\hypertarget{ref-olthofComplexityPsychologicalSelfratings2020}{}}%
Olthof, Merlijn, Fred Hasselman, and Anna Lichtwarck-Aschoff. 2020.
{``Complexity in Psychological Self-Ratings: Implications for Research
and Practice.''} \emph{BMC Medicine} 18 (1): 317.
\url{https://doi.org/10.1186/s12916-020-01727-2}.

\leavevmode\vadjust pre{\hypertarget{ref-stinsonEcologicalMomentaryAssessment2022}{}}%
Stinson, Lesleigh, Yunchao Liu, and Jesse Dallery. 2022. {``Ecological
{Momentary Assessment}: {A Systematic Review} of {Validity Research}.''}
\emph{Perspectives on Behavior Science} 45 (2): 469--93.
\url{https://doi.org/10.1007/s40614-022-00339-w}.

\leavevmode\vadjust pre{\hypertarget{ref-webber2005recurrence}{}}%
Webber Jr, Charles L, and Joseph P Zbilut. 2005. {``Recurrence
Quantification Analysis of Nonlinear Dynamical Systems.''}
\emph{Tutorials in Contemporary Nonlinear Methods for the Behavioral
Sciences} 94 (2005): 26--94.

\end{CSLReferences}



\end{document}
